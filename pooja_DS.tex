%%%%%%%%%%%%%%%%%
% This is an example CV created using altacv.cls (v1.3, 10 May 2020) written by
% LianTze Lim (liantze@gmail.com), based on the
% Cv created by BusinessInsider at http://www.businessinsider.my/a-sample-resume-for-marissa-mayer-2016-7/?r=US&IR=T
%
%% It may be distributed and/or modified under the
%% conditions of the LaTeX Project Public License, either version 1.3
%% of this license or (at your option) any later version.
%% The latest version of this license is in
%%    http://www.latex-project.org/lppl.txt
%% and version 1.3 or later is part of all distributions of LaTeX
%% version 2003/12/01 or later.
%%%%%%%%%%%%%%%%

%% If you are using \orcid or academicons
%% icons, make sure you have the academicons
%% option here, and compile with XeLaTeX
%% or LuaLaTeX.
% \documentclass[10pt,a4paper,academicons]{altacv}

%% Use the "normalphoto" option if you want a normal photo instead of cropped to a circle
% \documentclass[10pt,a4paper,normalphoto]{altacv}

\documentclass[10pt,a4paper,ragged2e,withhyper]{altacv}

%% AltaCV uses the fontawesome5 and academicon fonts
%% and packages.
%% See http://texdoc.net/pkg/fontawesome5 and http://texdoc.net/pkg/academicons for full list of symbols. You MUST compile with XeLaTeX or LuaLaTeX if you want to use academicons.

% Change the page layout if you need to
\geometry{left=1.25cm,right=1.25cm,top=1.5cm,bottom=1.5cm,columnsep=1.2cm}

% The paracol package lets you typeset columns of text in parallel
\usepackage{paracol}


% Change the font if you want to, depending on whether
% you're using pdflatex or xelatex/lualatex
\ifxetexorluatex
  % If using xelatex or lualatex:
  \setmainfont{Lato}
\else
  % If using pdflatex:
  \usepackage[default]{lato}
\fi

% Change the colours if you want to
\definecolor{VividPurple}{HTML}{3E0097}
\definecolor{SlateGrey}{HTML}{2E2E2E}
\definecolor{LightGrey}{HTML}{666666}
% \colorlet{name}{black}
\colorlet{tagline}{VividPurple}
\colorlet{heading}{VividPurple}
\colorlet{headingrule}{VividPurple}
% \colorlet{subheading}{PastelRed}
\colorlet{accent}{VividPurple}
\colorlet{emphasis}{SlateGrey}
\colorlet{body}{LightGrey}

% Change some fonts, if necessary
% \renewcommand{\namefont}{\Huge\rmfamily\bfseries}
% \renewcommand{\personalinfofont}{\footnotesize}
% \renewcommand{\cvsectionfont}{\LARGE\rmfamily\bfseries}
% \renewcommand{\cvsubsectionfont}{\large\bfseries}

% Change the bullets for itemize and rating marker
% for \cvskill if you want to
\renewcommand{\itemmarker}{{\small\textbullet}}
\renewcommand{\ratingmarker}{\faCircle}

%% sample.bib contains your publications
\addbibresource{sample.bib}

\begin{document}
\name{Pooja Gajare}
\tagline{Data Scientist}
% Cropped to square from https://en.wikipedia.org/wiki/Marissa_Mayer#/media/File:Marissa_Mayer_May_2014_(cropped).jpg, CC-BY 2.0
%% You can add multiple photos on the left or right
\photoR{2.7cm}{pooja.png}
% \photoL{2cm}{Yacht_High,Suitcase_High}
\personalinfo{%
  % Not all of these are required!
  % You can add your own with \printinfo{symbol}{detail}
  \email{pooja@gmx.net}
%   \phone{000-00-0000}
  \mailaddress{Breite strasse,9 13187}
  \location{Berlin, Germany}
% \homepage{marissamayr.tumblr.com}
%  \twitter{@marissamayer}
  \linkedin{pooja-gajare-a07bb7136}
   \github{pi-dotcom} % I'm just making this up though.
%   \orcid{orcid.org/0000-0000-0000-0000} % Obviously making this up too. If you want to use this field (and also other academicons symbols), add "academicons" option to \documentclass{altacv}
  %% You MUST add the academicons option to \documentclass, then compile with LuaLaTeX or XeLaTeX, if you want to use \orcid or other academicons commands.
  % \orcid{0000-0000-0000-0000}
  %% You can add your own arbtrary detail with
  %% \printinfo{symbol}{detail}[optional hyperlink prefix]
  % \printinfo{\faPaw}{Hey ho!}
  %% Or you can declare your own field with
  %% \NewInfoFiled{fieldname}{symbol}[optional hyperlink prefix] and use it:
  % \NewInfoField{gitlab}{\faGitlab}[https://gitlab.com/]
  % \gitlab{your_id}
}

\makecvheader

%% Depending on your tastes, you may want to make fonts of itemize environments slightly smaller
% \AtBeginEnvironment{itemize}{\small}

%% Set the left/right column width ratio to 6:4.
\columnratio{0.6}

% Start a 2-column paracol. Both the left and right columns will automatically
% break across pages if things get too long.
\begin{paracol}{2}

\cvsection{Profile}
\begin{Paragraph}
Enthusiastic Data Scientist eager to contribute to team success through hard work attention to detail and excellent organizational skills. Experience of Machine Learning and Data handling using python and SQL. Motivated to work, learn, grow and excel in organization.
\end{Paragraph}
\begin{comment}
\cvevent{President \& CEO}{Yahoo!}{July 2012 -- Ongoing}{Sunnyvale, CA}
\begin{itemize}
\item Led the \$5 billion acquisition of the company with Verizon -- the entity which believed most in the immense value Yahoo!\ has created
\item Acquired Tumblr for \$1.1 billion and moved the company's blog there
\item Built Yahoo's mobile, video and social businesses from nothing in 2011 to \$1.6 billion in GAAP revenue in 2015
\item Tripled the company's mobile base to over 600 million monthly active users and generated over \$1 billion of mobile advertising revenue last year
\end{itemize}
\end{comment}

\divider
\cvsection{Experience}
\cvevent{Intensive Data Science Bootcamp}{Spiced Academy}{Jan 2021 -- April 2021}{}
\begin{itemize}
\item 480 hours of Intensive Data Science Bootcamp Training.
\item Supervised and unsupervised machine learning.
\item Operating with data in SQL and NoSQL databases and building Dashboards.
\item Using python to collect, analyze data as well as data visualization focusing on the powerful scientific package libraries.
\item Deploying code on cloud servers using docker and AWS
\end{itemize}

\divider
\cvevent{Projects}{}{Jan 2021 -- April 2021}{}
\begin{itemize}

\item Classification (Titanic Dataset): Python, Pandas, Numpy, Matplotlib, Seaborn, Scikit-learn and Stats-model.
\item Dashboard: PostgreSQL, AWS RDS and  EC2, Metabase.
\item Data Pipeline: Docker, Tweepy API and Slack bot , MongoDB , PostgreSQL, ETL Jobs Airflow.
\item Classification of Cloths in Fashion-MNIST Dataset using Neural Network: Tensorflow-Keras
\item Recommender System: NMF-Model, Web development using Flask, CI/CD.
\item Android Apps Classification: Web Scrapping, Docker, ETL Jobs Airflow, NLP, Text-Classification with Transformers.
\end{itemize}

\divider

\cvevent{Design Engineer}{Neeraj Engineering}{September 2015 --  November 2017}{}
\begin{itemize}
    \item Supported engineering design development through analysis and simulation of prototype and 3D computer Models.
\item Involved in the tasks of planning execution and guidance of employees in area of Mechanical Design Development
\end{itemize}
% \divider

% \cvevent{Product Engineer}{Google}{23 June 1999 -- 2001}{Palo Alto, CA}

% \begin{itemize}
% \item Joined the company as employe \#20 and female employee \#1
% \item Developed targeted advertisement in order to use user's search queries and show them related ads
% \end{itemize}
\begin{comment}


\cvsection{A Day of My Life}

% Adapted from @Jake's answer from http://tex.stackexchange.com/a/82729/226
% \wheelchart{outer radius}{inner radius}{
% comma-separated list of value/text width/color/detail}
% Some ad-hoc tweaking to adjust the labels so that they don't overlap
\hspace*{-1em}  %% quick hack to move the wheelchart a bit left
\wheelchart{1.5cm}{0.5cm}{%
  10/13em/accent!30/Sleeping \& dreaming about work,
  25/9em/accent!60/Public resolving issues with Yahoo!\ investors,
  5/11em/accent!10/\footnotesize\\[1ex]New York \& San Francisco Ballet Jawbone board member,
  20/11em/accent!40/Spending time with family,
  5/8em/accent!20/\footnotesize Business development for Yahoo!\ after the Verizon acquisition,
  30/9em/accent/Showing Yahoo!\ \mbox{employees} that their work has meaning,
  5/8em/accent!20/Baking cupcakes
}


% use ONLY \newpage if you want to force a page break for
% ONLY the currentc column
\newpage

\cvsection{Publications}

\nocite{*}

\printbibliography[heading=pubtype,title={\printinfo{\faBook}{Books}},type=book]

\divider

\printbibliography[heading=pubtype,title={\printinfo{\faFile*[regular]}{Journal Articles}}, type=article]

\divider

\printbibliography[heading=pubtype,title={\printinfo{\faUsers}{Conference Proceedings}},type=inproceedings]
\end{comment}
%% Switch to the right column. This will now automatically move to the second
%% page if the content is too long.
\switchcolumn

\cvsection{Life Philosophy}
\begin{quote}
``Fight until its not the end that you had imagined.''
\end{quote}

\cvsection{Skills}

\cvachievement{\faPython}{Programming Languages}{Python}
\cvachievement{\faDatabase}{Databases}{SQL, MongoDB, CouchDB}
\cvachievement{\faBrain}{AI}{Machine Learning, Deep Learning, Tensorflow, Scikit-learn}
\cvachievement{\faDocker}{Docker}{docker-compose}
\cvachievement{\faAws}{AWS Services}{}
\cvachievement{\faColumns}{Dashboard}{Metabase}
\cvachievement{\faCube}{3D Modelling}{Autocad, CATIA V5}

\cvsection{Soft Skills}
\begin{itemize}
\item Communication
\item Problem Solving
\item Teamwork/Collaboration
\item Networking
\item Proactive
\end{itemize}

\cvsection{Languages}

\cvskill{English}{5}
% \divider

\cvskill{German}{3}

\cvskill{Hindi}{5}

\cvskill{Marathi}{5}

% \divider


\cvsection{Education}
\cvevent{Bachelors in Mechanical Engineering}{Savitribai Phule Pune University, Pune, India}{}{}

\divider

\cvevent{Diploma in Mechanical Engineering}{MSBTE Mumbai University,  India}{}{}

\begin{comment}



\newpage

\cvsection{Referees}

% \cvref{name}{email}{mailing address}
\cvref{Prof.\ Alpha Beta}{Institute}{a.beta@university.edu}
{Address Line 1\\Address line 2}

\divider

\cvref{Prof.\ Gamma Delta}{Institute}{g.delta@university.edu}
{Address Line 1\\Address line 2}
\end{comment}
\end{paracol}

\end{document}
